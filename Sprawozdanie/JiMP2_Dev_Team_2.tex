\documentclass[11pt]{article}
\usepackage[polish]{babel}
\usepackage[T1]{fontenc}
\usepackage[utf8]{inputenc}

\begin{document}

\begin{huge}
\begin{center}
\textbf{Automat komórkowy w języku C}\\
\end{center}
\end{huge}

\section{Wstęp}
Automat komórkowy jest systemem składającym się z pojedynczych komórek sąsiadujących ze sobą. Każda komórka przyjmuje jeden ze skończonego zbioru stanów. Komórka zmienia swój stan na podstawie przyjętych założeń, opierających się na stanie komórek z nią sąsiadujących. Takim przykładem automatu komórkowego jest gra w życie. Jej zasady to:
Jeśli żywa komórka ma 2 lub 3 żywych sąsiadów to pozostaje żywa, w przeciwnym wypadku umiera. Jeśli martwa komórka ma 3 żywych sąsiadów to ozywa, inaczej pozostaje martwa.
W naszej grze wykorzystywane jest sąsiedztwo Moore’a, które zakłada sąsiedztwo 8 komórek wokół komórki badanej.


\section{Kompilacja kodu oraz wywołanie programu}

W naszym projekcie przyjęliśmy, że przy wywoływaniu programu podawany jest plik o przykładowym wyglądzie: 
\\
3 4 
\\
1 0 0
\\
1 1 1
\\
1 1 1
\\
0 0 0
\\
W pierwszym wierszu pliku mamy wymiary planszy. Liczba 3 jest liczbą kolumn, natomiast liczby 4 jest liczbą wierszy naszej tablicy. Następne wiersze to już sama plansza, w której 0 reprezentują komórki martwe, a 1 komórki żywe.Aby wywołać program otwieramy terminal, a następnie wchodzimy w folder, w którym program się znajduje. Do terminalu wpisujemy komendę \textit{make}, która korzystając z reguł zawartych w pliku Makefile kompiluje nasz kod oraz tworzy program $GameofLife$.Następnie wywołujemy program z wybranym przez nas plikiem, np.\\
$./GameofLife$ $twojplikzdanymi.txt.$
\\
Program przechowuje planszę w tablicy dwuwymiarowej. Tworzone są 2 tablice. Jedna z nich używana jest do wprowadzania zmian w generacji, a druga służy do porównywania sąsiedztwa komórki w celu wprowadzenia zmian w pierwszej tablicy.

\section{Funkcjonowanie programu}

Kolejne etapy działania programu:
\\
1.	Pobranie danych z pliku i zapisanie ich do tablic.
\\
2.	Spytanie o ilość generacji i zapisanie tej danej.
\\
3.	Wygenerowanie nazwy pliku do generacji.
\\
4.	Porównanie komórek i wygenerowanie nowej planszy.
\\
5.  Utworzenie pliku *.png z wygenerowaną uprzednio nazwą i zapisanie do niego planszy w postaci komórka żywa - kolor biały, komórka martwa - kolor czarny.
\\
Plik Makefile, o którym była mowa wcześniej zawiera 2 reguły, które kompilują kod zawarty w w plikach main.c oraz funkcje.c z flagą -lpng, aby dołączyć bibliotekę libpng. Korzystamy również z utworzonego przez nas pliku nagłówkowego "funkcje.h".
\\
Wynikiem działania programu są pliki nazwie Gen[x].png , gdzie x jest numerem generacji.


\end{document}