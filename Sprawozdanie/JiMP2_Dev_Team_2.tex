

\documentclass[11pt]{article}
\usepackage[polish]{babel}
\usepackage[T1]{fontenc}
\usepackage[utf8]{inputenc}

\begin{document}

\begin{huge}
\begin{center}
\textbf{Automat kom\'orkowy w j\k{e}zyku C}\\
\end{center}
\end{huge}

\section{Wst\k{e}p}
Automat komórkowy jest systemem składającym się z pojedynczych komórek sąsiadujących ze sobą. Każda komórka przyjmuje jeden ze skończonego zbioru stanów. Komórka zmienia swój stan na podstawie przyjętych założeń, opierających się na stanie komórek z nią sąsiadujących. Takim przykładem automatu komórkowego jest gra w życie. Jej zasady to:
Jeśli żywa komórka ma 2 lub 3 żywych sąsiadów to pozostaje żywa, w przeciwnym wypadku umiera. Jeśli martwa komórka ma 3 żywych sąsiadów to ozywa, inaczej pozostaje martwa.
W naszej grze wykorzystywane jest sąsiedztwo Moore’a, które zakłada sąsiedztwo 8 komórek wokół komórki badanej.


\section{Program}

W naszym projekcie przyjęliśmy, że przy wywoływaniu programu podawany jest plik o przykładowym wyglądzie:

3 4\\
1 0 0\\
1 1 1\\
1 1 1\\
0 0 0\\
\\
Pierwsza liczba (3) to liczba kolumn, druga (4) rzędów –są to wymiary planszy. Dalej podajemy wygląd planszy składający się z 0 (komórki martwe) i 1 (komórki żywe). Cyfry w jednym rzędzie oddzielane są spacjami,a rzędy enterami. Aby wywołać program otwieramy terminal, a nastepnie wchodzimy w folder, w którym program sie znajduje. Do teminala wpisujemy 'make' aby przekompilować program. Potem wpisujemy: ./Game_of_Life twoj_plik_z_danymi.txt. 
\\
\\
Program przechowuje planszę w tablicy dwuwymiarowej. Tworzone są 2 tablice. Jedna używana jest do wprowadzania zmian w generacji, a druga służy do porównywania sąsiedztwa komórki w celu wprowadzenia zmian w pierwszej tablicy.
\\
\\
Kolejne etapy działania programu:\\
1.	Pobranie danych z pliku i zapisanie ich do tablic.\\
2.	Spytanie o ilość generacji i zapisanie tej danej.\\
3.	Wygenerowanie nazwy pliku do generacji.\\
4.	Porównanie komórek i wygenerowanie nowej planszy.

\end{document}
